\documentclass[a4paper,11pt,twoside]{article}
%\documentclass[a4paper,11pt,twoside,se]{article}

\usepackage{UmUStudentReport}
\usepackage{verbatim}   % Multi-line comments using \begin{comment}
\usepackage{courier}    % Nicer fonts are used. (not necessary)
\usepackage{pslatex}    % Also nicer fonts. (not necessary)
\usepackage[pdftex]{graphicx}   % allows including pdf figures
\usepackage{listings}
\usepackage{pgf-umlcd}
\usepackage{blindtext}
\usepackage{enumitem}
%\usepackage{lmodern}   % Optional fonts. (not necessary)
%\usepackage{tabularx}
%\usepackage{microtype} % Provides some typographic improvements over default settings
%\usepackage{placeins}  % For aligning images with \FloatBarrier
%\usepackage{booktabs}  % For nice-looking tables
%\usepackage{titlesec}  % More granular control of sections.

% DOCUMENT INFO
% =============
\department{Department of Computing Science}
\coursename{Application Development in Java 7.5 p}
\coursecode{5DV135}
\title{Self Reflection Project Work}
\author{Lorenz Gerber ({\tt{dv15lgr@cs.umu.se}} {\tt{lozger03@student.umu.se}})}
\date{2017-01-11}
%\revisiondate{2016-01-18}
\instructor{Johan Eliasson / Jan Erik Moström / Alexander Sutherland / Filip Allberg / Adam Dahlgren Lindström}


% DOCUMENT SETTINGS
% =================
\bibliographystyle{plain}
%\bibliographystyle{ieee}
\pagestyle{fancy}
\raggedbottom
\setcounter{secnumdepth}{2}
\setcounter{tocdepth}{2}
%\graphicspath{{images/}}   %Path for images

\usepackage{float}
\floatstyle{ruled}
\newfloat{listing}{thp}{lop}
\floatname{listing}{Listing}



% DEFINES
% =======
%\newcommand{\mycommand}{<latex code>}

% DOCUMENT
% ========
\begin{document}
\lstset{language=C}
\maketitle
\thispagestyle{empty}
\newpage
\tableofcontents
\thispagestyle{empty}
\newpage

\clearpage
\pagenumbering{arabic}

\section{Introduction}
This report is a short self reflection about the group programming project in Java application programming course. The project lasted about one month. Our group consisted of four persons, all from the DV undergraduate program. The oral presentation was a short end of project presentation of the finished product. The specification of the presentation are given in detail on the course homepage.

\section{Project Work General}
I am in general satisfied with my performance for project work in this specific project. I was sticking to the schedule that was decided in the beginning and during the project I felt that I can judge quite reliably the time individual tasks will take for me to finish.
My current situation is that I have a 50\% job as a programmer, remote. While I would have the flexibility to adapt my working hours, I find it rather difficult. It's easier for me to have a daily routine which usually goes like University in the morning, work in the afternoon and additional study time in the evening. This was also the mode during this project. However, it happened rather often in the afternoon, that I anyway switched to the `App Java' project as I knew that the others are working with it. Or even when I could see that new commits drop in on the repo. I realized that this might be a difficulty for me: It is always tempting to spend more time on work that pays off immediately. In terms of this project, we advanced very quick with enjoyable visual and audible rewards through our really cool game. So it was easy to work on it, while my pay job is sometimes a bit more of a long shot until I get a `reward' in the form of a new implemented feature (that is, based on the type of software we develop, still not as flashy as a computer game).
I don't think that discipline is my weak side, but I can definitively see how it is an ongoing process where one always has to work to stay efficient. 

\section{Teamwork in Project}
Teamwork in this project was a blast: We had so much fun and even organized two dinner evenings for the whole group. It was nice to see how our different skills and interests formed a team spirit and made me lookforward to actually go to university and sit together with the group instead of coding from home remote and communicate on chat.
The most striking thing that I realized was an obvious weakness of mine when it comes to taking the project manager role in group work. We were lucky enough to have one student that has already a degree in IT management. It was therefore an obvious choice to assign this student as project manager. This turned out interesting for me, because otherwise, I often happen to end up in this role, probably because I am rather proactive and have easy to communicate with others. However, I also often feel a large burden as project leader when it comes to assigning tasks to team members. Now I had the possibility to see this role from the outside. When I'm project  manager, I usually put down a lot of thought not just to what I should do but also what all the others should/could do in the project. In this project, the project manger simply listed the open tasks and assigned them sequentially, without any preferences on all team members. When I saw this, I realized that I often spend too much time on figuring out who could do what instead of simply assigning so that everybody can start to work. If problems show up, tasks can still be shifted amongst team members. I'm curious how this insight will turn out for me when I will have the project manager role next time.
Otherwise, I can just say that everyone in our group went in 200\% for the project and I had the impression that everyone really enjoyed that feeling of being a little bit closer now to how `real-world' industry IT project work really happens.  

\section{Oral Presentation}
I have generally little problems to do oral presentations and get usually decent feedback. This is maybe also due to my former job as a researcher where I often was giving lectures, research presentations or spoke at journal clubs. For me, usually oral presentations are a bit of a freestyle thing. I don't prepare presentations into the last detail and try to react then also a bit on the crown or how the time situation plays out. Now the freestyle thing is a bit of a disadvantage when doing the presentation together with three other presenters that might have completely different preferences. I realized that for the project presentation, I probably had a bit a bad influence to the others as I tend tone down the need of too rigorous preparations for the presentation. On the other side, I also felt that the requirements for the oral presentation given on the homepage were rather sloppy, not very detailed.
However, I still think that we performed reasonably for our presentation. My largest handicap was that only two days before the presentation, I completely lost my voice due to a cold. This was however good, as there were some team members that are maybe a bit less comfortable with speaking in front of crowds. Like this, they were forced to take a bit a larger share of the presentation and they did well.


\addcontentsline{toc}{section}{\refname}
\bibliography{references}

\end{document}
